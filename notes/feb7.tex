\documentclass[10pt]{article}
\textheight=9.25in \textwidth=7in \topmargin=-.75in
 \oddsidemargin=-0.25in
\evensidemargin=-0.25in
\usepackage{url}  % The bib file uses this
\usepackage{graphicx} %to import pictures
\usepackage{amsmath, amssymb, bbold}
\usepackage{theorem, multicol, color}
\usepackage{gfsartemisia-euler} % best font in da game
\usepackage{tikz} % Graphs and other graphics

\setlength{\intextsep}{5mm} \setlength{\textfloatsep}{5mm}
\setlength{\floatsep}{5mm}
\setlength{\parindent}{0em} % new paragraphs are not indented
\setcounter{MaxMatrixCols}{20}
\usepackage{caption}
\captionsetup[figure]{font=small}


%%%%  SHORTCUT COMMANDS  %%%%
\newcommand{\ds}{\displaystyle}
\newcommand{\Z}{\mathbb{Z}}
\newcommand{\1}{\mathbb{1}}
\newcommand{\arc}{\rightarrow}
\newcommand{\R}{\mathbb{R}}
\newcommand{\N}{\mathbb{N}}
\newcommand{\Q}{\mathbb{Q}}
\renewcommand{\P}{\mathbb{P}}
\newcommand{\E}{\mathbb{E}}
\newcommand{\blank}{\underline{\hspace{0.33in}}}
\newcommand{\qand}{\quad and \quad}
\renewcommand{\stirling}[2]{\genfrac{\{}{\}}{0pt}{}{#1}{#2}}
\newcommand{\dydx}{\ds \frac{d y}{d x}}
\newcommand{\ddx}{\ds \frac{d}{d x}}
\newcommand{\dvdx}{\ds \frac{d v}{d x}} 
\renewcommand{\part}{\partial}
%%%%  footnote style %%%%

\renewcommand{\thefootnote}{\fnsymbol{footnote}}

\pagestyle{empty}

\begin{document}

\begin{flushright}
Chandler Justice - A02313187
\end{flushright}
\noindent \underline{\hspace{3in}}\\

\textbf{February 7, 2024}\\

\underline{revisiting the heat equation}

\[\begin{cases}
\frac{\part u}{\part t} = \frac{\part}{\part x} (K \frac{\part u}{\part t} + 4(\gamma t)\\
u(x, 0) = g(x)\\
u(a, t) = \alpha(t)\\
u(b, t) = \beta(t)
\end{cases}\]

\[\begin{cases}
u'' = f \quad x \in (0,1)\\
u(0) = \alpha\\
u(1) = \beta\\
\end{cases} \Rightarrow
u'' = \frac{u(x+h) -2u(x) + u(x-h)}{h^2}\]

\[\frac{U_{j-1} - 2U_j + U_{j+1}}{h^2} = f(x_j) \Rightarrow U_j \approx u(\alpha i) \Rightarrow A^h U^h = F^h\]
Where $A$ is the fucking tridiagonal matrix, U is the approximation at a given value, and $F$ is the result of $U^h * A^h$.

\[\therefore U^h = (A^h)^{-1} F^h\]
This gives approximations at
\[M_h = \{x_1,x_2,...,x_m\}, \quad h = \text{mesh size}\]

\underline{Newmann condition}

\[\begin{cases}
u'' = f\\
u'(0) = \sigma \Rightarrow (U_0) = \sigma\\
u(1) = \beta
\end{cases}\]

We still set up a mesh
\[M_k = \{x_0, x_1, x_2, ..., x_k\}, \quad h = 1/m+1\]

For $j = 0$:

\[\frac{U_{j-1} - 2U_j + U_{j+1}}{h^2} = \frac{U_{-1} - 2U_0 + U_1}{h^2}\]

The idea is to modify the difference quotient for $j=0$.\\

For $j = 1$: 

\[\frac{U_{0} - 2U_1 + U_{2}}{h^2} = f(x_j)\]

\textit{modifying differnce for $j = 0$}\\
\textbf{Method 1:}\\
\[u'(0) = \beta \approx \frac{U_1 - U_0}{h}\]
\[A^hU^h = 1/h^2 \begin{bmatrix}
    -h & h & ... & ... & ...\\
    1 & -2 & 1 & 0 & ... & 0\\
    0 & 1 & -2 & 1 & .... & 0\\
    ... & ... & ... & ... & ...\\
    0 & 0 & ... & 1 & -2 & 1\\
    0 & ... & ... & ... & 0 & h^2
\end{bmatrix}
\begin{bmatrix}
    U_0\\
    U-1\\
    ...\\
    U_m\\
    U_{m+1}
\end{bmatrix}
=
\begin{bmatrix}
\sigma\\
f(x_1)\\
...\\
f(x_{m-k})\\
\beta
\end{bmatrix}
\]

From we can analyze
\[|u'(0) - \frac{u(h) - u(0)}{h}| \leq Ch'\]
This forces all local truncation errors to be first order accurate, even though the central differences are more accurate.\\

\textbf{Method 2:} introduce fictitious nodes\\
This requires extrapolating outside of our domain, which we do not generally want to do.
\[u'(0) = \sigma \approx \frac{U_1 - U_{-1}}{2h} \quad \text{LTE at } \bar{x} = 0 \rightarrow \text{2nd order accurate}\]
\[\frac{U_1 - U_{-1}}{2h} = \sigma \Rightarrow U_{1} - 2h\sigma = U_{-1}\]
From here, we solve for the fictitious node $U_{-1}$.\\

\textbf{Method 3:} Interpolate within\\
Try to take $0, h, 2h$ and interpolate a derivative within desired range.

\[\frac{1}{h} \left(\frac{3}{2} U_0 - 2U_1 + \frac{1}{2}U_2\right) = \sigma\]

Using this method, the structure of the matrix remains the same,
\[ 1/h^2
    \begin{bmatrix}
        \frac{3}{2} h & -2h & \frac{1}{2}h & ... & 0 & 0\\
        1 & -2 & 1 & ... & 0 & 0\\
        0 & 1 & -2 & 1 & ... & 0\\
        ...\\
    \end{bmatrix}
\begin{bmatrix}
U_0\\
U_1\\
...\\
U_m\\
U_{m+1}
\end{bmatrix}\]
\newpage
\textbf{February 9, 2024}\\

\underline{Reordering of nodes}\\

\[ 1/h^2
    \begin{bmatrix}
        \frac{3}{2} h & -2h & \frac{1}{2}h & ... & 0 & 0\\
        1 & -2 & 1 & ... & 0 & 0\\
        0 & 1 & -2 & 1 & ... & 0\\
        ...\\
    \end{bmatrix}
\begin{bmatrix}
U_0\\
U_1\\
...\\
U_m\\
U_{m+1}
\end{bmatrix}
=
\begin{bmatrix}
    f(x_1) - \frac{\alpha}{h^2}\\
    f(x_2)\\
    f(x_3)\\
    ...\\
    f(x_{m-1})\\
    f(x_m) - \frac{\beta}{h^2}
\end{bmatrix}
\]

We can reorder things in terms of odd terms first and then even terms after.\\

\begin{align*}
    j &= 1\\
      &= \frac{1}{h^2}(- 2U_1 + U_2) = f(x_1) - \frac{\alpha}{h^2}\\
    j &= 3\\
      &=  \frac{1}{h^2}(U_2 - 2U_3 + U_4) = f(x_3) \\
    j &= 5\\
      &=  \frac{1}{h^2}(U_4 - 2U_5 + U_6) = f(x_3) \\
      ...\\
    j &= m-1\\
      &= \frac{1}{h^2}(U_{m-2} - 2U_{m-1} + U_m) = f(x_{m-1}) - \frac{\alpha}{h^2}\\
      &...\\
    j &= 2\\
      &=  \frac{1}{h^2}(U_1 - 2U_2 + U_3) = f(x_2) \\
      &...\\
    j &= m\\
      &= \frac{1}{h^2}(U_m-1 - 2U_{m}) = f(x_{m}) - \frac{\beta}{h^2}\\
\end{align*}

Which results in a matrix following the shape
\[1/h^2 =
    \begin{bmatrix}
    -2 & 0 & 0 & 1 & 0 & 0\\
    0 & -2 & 0 & 1 & 1 & 0\\
    0 & 0 & -2 & 0 & 1 & 1\\
    1 & 1 & 0 & -2 & 0 & 0\\
    0 & 1 & 1 & 0 & -2 & 0\\
    0 & 0 & 1 & 0 & 0 & -2\\
\end{bmatrix}
=
\begin{bmatrix}
    U_1\\
    U_3\\
    ...\\
    U_{m-1}\\
    U_2\\
    ...\\
    U_M
\end{bmatrix}
=
\begin{bmatrix}
    f(x_1) - \alpha/h^2\\
    f(x_3)\\
    ...\\
    f(x_{m-1})\\
    f(x_2)\\
    ...\\
    f(x_m)
\end{bmatrix}
\]
This divides the matrix into 4 equal regions.\\
\noindent \underline{\hspace{3in}}\\
\end{document}

