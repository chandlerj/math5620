\documentclass[10pt]{article}
\textheight=9.25in \textwidth=7in \topmargin=-.75in
 \oddsidemargin=-0.25in
\evensidemargin=-0.25in
\usepackage{url}  % The bib file uses this
\usepackage{graphicx} %to import pictures
\usepackage{amsmath, amssymb}
\usepackage{theorem, multicol, color}
\usepackage{gfsartemisia-euler}

\setlength{\intextsep}{5mm} \setlength{\textfloatsep}{5mm}
\setlength{\floatsep}{5mm}
\setlength{\parindent}{0em} % new paragraphs are not indented


%%%%  SHORTCUT COMMANDS  %%%%
\newcommand{\ds}{\displaystyle}
\newcommand{\Z}{\mathbb{Z}}
\newcommand{\arc}{\rightarrow}
\newcommand{\R}{\mathbb{R}}
\newcommand{\N}{\mathbb{N}}
\newcommand{\Q}{\mathbb{Q}}
\renewcommand{\P}{\mathbb{P}}
\newcommand{\blank}{\underline{\hspace{0.33in}}}
\newcommand{\qand}{\quad and \quad}
\renewcommand{\stirling}[2]{\genfrac{\{}{\}}{0pt}{}{#1}{#2}}
\newcommand{\dydx}{\ds \frac{d y}{d x}}
\newcommand{\ddx}{\ds \frac{d}{d x}}
\newcommand{\dvdx}{\ds \frac{d v}{d x}} 

%%%%  footnote style %%%%

\renewcommand{\thefootnote}{\fnsymbol{footnote}}

\pagestyle{empty}

\begin{document}

\begin{flushright}
Chandler Justice - A02313187
\end{flushright}
\noindent \underline{\hspace{3in}}\\

\textbf{January 19, 2024}\\

\begin{center}
        \textbf{Vandermonde matrix}
\end{center}
\[\begin{bmatrix}
    1 & 1 & 1 & ... & 1\\
    (x_1 - \bar{x}) & (x_2 - \bar{x}) & (x_3 - \bar{x}) & ... & (x_n - \bar{x})\\
    (x_1 - \bar{x})^2 & (x_2 - \bar{x})^2 & (x_3 - \bar{x})^2 & ... & (x_n - \bar{x})^2\\
    ... & ... & ... & ... & ...\\
    (x_1 - \bar{x})^n & (x_2 - \bar{x})^n & (x_3 - \bar{x})^n & ... & (x_n - \bar{x})^n\\
\end{bmatrix}
\begin{bmatrix}
c_1\\
c_2\\
...\\
c_n\\
\end{bmatrix}
=
\begin{bmatrix}
0\\
1\\
...\\
0\\
\end{bmatrix}\]
\begin{itemize}
    \item We can move the $1$ in the matrix in the vector on the RHS to compute different derivatives.
    \item $n > k+1$ is required to get a good approximation of the derivative
\end{itemize}


\textbf{Example:} $\ds D_0 u(\bar{x}) = \frac{u(x + h) - u(x-h)}{2h}$\\

Let $k = 1$, then we rewrite our approximation as
\[C_1 u(x + h) + C_0 (x) + C_{-1}a(x-h)\]
Now that we have rephrased our approximation, we can see $n = 3$ (where $n$ is the number of coefficients)\\

\textbf{Chapter 2: Building methods for solving DEs}\\

\textbf{Example:} Heat equation. Consider the flow of heat in a 1D rod of material that conducts heat. What happens as time progresses with the rod in regards to its temperature. Let the length of the rod be $L = 1$.

The DE used to heat in relation to time is
\[\frac{\partial u}{\partial t} = \frac{\partial}{\partial x}(k(x) \frac{\partial u}{\partial x} = f(x)) \]

If $k$ is constant, then

\[\frac{\partial u}{\partial t} = k \frac{\partial^2 u}{\partial x^2} + f(x)\]

\underline{Steady state}
\[\frac{\partial u}{\partial t} = 0\]
\[\Rightarrow k\frac{\partial^2 u}{\partial x^2} - f(x) = 0\]
\[\frac{d^2 u}{dx^2} = f(x) \Rightarrow u'' = f\]
\[\Rightarrow u(0) = \alpha, u(1) = \beta\]

\underline{Structured Stability}
\[\begin{cases}
    y'' + \frac{P}{EI} y = 0\\
    y(0) = 0\\
    y(c) = 0\\
\end{cases}
\]
this problem has $\infty$ solutions, so it is a bad idea to use numerical methods on this DE. We will want to use finite difference methods to get around this.
\newpage
\underline{Finite Methods}
\[u'' = f(x) \rightarrow u' = \int f(x)dx + C_1 = g(x) + C_1\]
\[u = \int g(x) dx + C_1x + C_2\]

\[u'' = \frac{u(x + h) -2a(x) + u(x + h)}{h^2} + E\]

Need some points for our linear approx: take equally spaced points on $[0, 1]$
\[h = \frac{1-0}{m} \Rightarrow h = 1/m\]
\noindent \underline{\hspace{3in}}\\

\end{document}

